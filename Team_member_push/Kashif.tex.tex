\documentclass[12pt,a4paper]{article}
\usepackage[utf8]{inputenc}
\usepackage[margin=1in]{geometry}
\usepackage{hyperref}
\usepackage{enumitem}

\title{Stakeholder Identification, Tools and Technologies}
\author{}
\date{}

\begin{document}

\section*{Stack holder identification:}

\begin{itemize}
    \item \textbf{Freelancer(enduser):} As the project iss about developing template for the user specially for freelancers so its end user is a freelancer that may interact this software to create proposals for thier client. This system will provide template to end user so they could make proposal easily for their client. The end user can also analyze their client profile and can also generate personalized proposals based on different templates provided by the software.
    
    \item \textbf{Administrator:} The system administrators are also stack holder for this software as he will be responsible for managing the functionality of their software. They can also check the profiles of the system users. They is also responsible for the ethical use of the system, if there is any unethical activity is being held they can report to the legislative structure. In case of managing system, the administrator is responsible to report the bugs, updating and adding new features in the system to developers.
    
    \item \textbf{Developers:} Developers are also user for this software as they are responsible for the smooth functionality and can use according to the constraints of the administrator in case of adding, removing and updating features. If there is any error occurs in the system after the launching, they are responsible to check this bug and remove it. In this way, they are the salient users of the system.
    
    \item \textbf{Project supervisors:} are also the user of the software as they are responsible to provide a good featured software to the investors, system administrator and other key figure of the project. They works as a bridge between the developers team and administrative team. So they is also key user for the software.
    
    \item \textbf{Investors/external platform:} such as upwork can also use this software as the part of their own to provide facility to their users.
\end{itemize}

So the system could have multiple users, so we can say that system could be used publically.

\section*{Tools and Technologies:}

Developing software is not an easy process, there require a lot of requirement.

To implement these requirement, we use different tools. In this system, we are going to use lot of tools in case of documentation and designing. Some of the tools are as follows

\textbf{Latex:} As latex is high quality documents preparation tool so we will be writing proposals and requirements from the RP on latex. All type of documentation will be prepared through this tool

\textbf{GitHub:} We will be using GitHub for the storing of proposals by creating repositories and for the collaboration between the projects members. We will also be storing meeting minute and videos that will be captured during the meeting with RP. We will also be using GitHub for the controlling of version.

\textbf{Figma:} Figma is powerful tool for creating layout for the software. We can create interactive prototype for our system.

\textbf{Canva:} we will be aslo using this tool to create design for our system. Some graphics will also being done here. So Canva will also be used.

\textbf{LucidChart:} This is an advanced UML which provide us different feature for the creating flowchart to present requirement visually. This toll will also be used to manage requirement and creating simplicity.

\section*{References:}

Here are some link of sites from where the help for the developing of this proposal has been taken.

\begin{itemize}[label={}]
    \item \url{https://mentorsol.com/best-software-development-tools/}
    \item \url{https://www.upwork.com/resources/how-to-create-a-proposal-that-wins-jobs}
    \item \url{https://pmc.ncbi.nlm.nih.gov/articles/PMC8021201/}
    \item \url{https://www.mdpi.com/2076-3417/12/21/10698}
\end{itemize}

\end{document}